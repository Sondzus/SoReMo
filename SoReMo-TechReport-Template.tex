% Options for packages loaded elsewhere
\PassOptionsToPackage{unicode}{hyperref}
\PassOptionsToPackage{hyphens}{url}
\PassOptionsToPackage{dvipsnames,svgnames*,x11names*}{xcolor}
%
\documentclass[
]{article}
\usepackage{lmodern}
\usepackage{amssymb,amsmath}
\usepackage{ifxetex,ifluatex}
\ifnum 0\ifxetex 1\fi\ifluatex 1\fi=0 % if pdftex
  \usepackage[T1]{fontenc}
  \usepackage[utf8]{inputenc}
  \usepackage{textcomp} % provide euro and other symbols
\else % if luatex or xetex
  \usepackage{unicode-math}
  \defaultfontfeatures{Scale=MatchLowercase}
  \defaultfontfeatures[\rmfamily]{Ligatures=TeX,Scale=1}
\fi
% Use upquote if available, for straight quotes in verbatim environments
\IfFileExists{upquote.sty}{\usepackage{upquote}}{}
\IfFileExists{microtype.sty}{% use microtype if available
  \usepackage[]{microtype}
  \UseMicrotypeSet[protrusion]{basicmath} % disable protrusion for tt fonts
}{}
\makeatletter
\@ifundefined{KOMAClassName}{% if non-KOMA class
  \IfFileExists{parskip.sty}{%
    \usepackage{parskip}
  }{% else
    \setlength{\parindent}{0pt}
    \setlength{\parskip}{6pt plus 2pt minus 1pt}}
}{% if KOMA class
  \KOMAoptions{parskip=half}}
\makeatother
\usepackage{xcolor}
\IfFileExists{xurl.sty}{\usepackage{xurl}}{} % add URL line breaks if available
\IfFileExists{bookmark.sty}{\usepackage{bookmark}}{\usepackage{hyperref}}
\hypersetup{
  pdftitle={Technical report template document},
  pdfauthor={FirstName LastName, Illinois Institute of Technology},
  colorlinks=true,
  linkcolor=darkblue,
  filecolor=Maroon,
  citecolor=Blue,
  urlcolor=darkblue,
  pdfcreator={LaTeX via pandoc}}
\urlstyle{same} % disable monospaced font for URLs
\usepackage[margin=1in]{geometry}
\usepackage{graphicx,grffile}
\makeatletter
\def\maxwidth{\ifdim\Gin@nat@width>\linewidth\linewidth\else\Gin@nat@width\fi}
\def\maxheight{\ifdim\Gin@nat@height>\textheight\textheight\else\Gin@nat@height\fi}
\makeatother
% Scale images if necessary, so that they will not overflow the page
% margins by default, and it is still possible to overwrite the defaults
% using explicit options in \includegraphics[width, height, ...]{}
\setkeys{Gin}{width=\maxwidth,height=\maxheight,keepaspectratio}
% Set default figure placement to htbp
\makeatletter
\def\fps@figure{htbp}
\makeatother
\setlength{\emergencystretch}{3em} % prevent overfull lines
\providecommand{\tightlist}{%
  \setlength{\itemsep}{0pt}\setlength{\parskip}{0pt}}
\setcounter{secnumdepth}{-\maxdimen} % remove section numbering

\title{Technical report template document}
\usepackage{etoolbox}
\makeatletter
\providecommand{\subtitle}[1]{% add subtitle to \maketitle
  \apptocmd{\@title}{\par {\large #1 \par}}{}{}
}
\makeatother
\subtitle{Spring/Fall 20XX SoReMo Fellowship Project Final Technical Report}
\author{FirstName LastName, Illinois Institute of Technology\footnote{\href{mailto:YOUREMAILHERE@hawk.iit.edu}{Contact
  the author}.}}
\date{}

\begin{document}
\maketitle

{
\hypersetup{linkcolor=}
\setcounter{tocdepth}{1}
\tableofcontents
}
\hypertarget{abstract}{%
\section{Abstract}\label{abstract}}

Write your abstract in plain English. Why did you do this project? What
did it accomplish?

\hypertarget{background-or-introduction}{%
\section{Background or introduction}\label{background-or-introduction}}

Tell us more about the \emph{context} of your project. Make sure to cite
other work on which you're building. Line breaks do not matter unless
there is a blank line between two lines of text - then it's a new
paragraph,

like it just happened here.

\hypertarget{another-section-such-as-technical-details}{%
\section{Another section, such as technical
details}\label{another-section-such-as-technical-details}}

What exactly did you do? Data, figures, etc.

\hypertarget{here-is-a-subsection}{%
\subsection{Here is a subsection}\label{here-is-a-subsection}}

you can focus on a specific aspect.

\begin{quote}
Here is a quote.
\end{quote}

\hypertarget{here-is-a-sub-subsection-with-lists}{%
\subsubsection{Here is a sub-subsection with
lists}\label{here-is-a-sub-subsection-with-lists}}

And in it I want to include a list:

\begin{itemize}
\tightlist
\item
  item 1
\item
  item 2
\item
  but there often needs to be a blank line before a bulleted list, for
  Markdown to recognize it and format as a list!,
\end{itemize}

or a numbered, nested one:

\begin{enumerate}
\def\labelenumi{\arabic{enumi})}
\tightlist
\item
  i like
\item
  lists so much

  \begin{itemize}
  \tightlist
  \item
    but they have to be
  \item
    easy to read.

    \begin{itemize}
    \tightlist
    \item
      indentation does the trick.
    \end{itemize}
  \end{itemize}
\end{enumerate}

\hypertarget{future-work}{%
\section{Future work}\label{future-work}}

What might be a cool followup to this project? how would you imagine its
impact?

\hypertarget{executive-summary-or-action-proposal}{%
\section{Executive summary, or action
proposal}\label{executive-summary-or-action-proposal}}

Specific things you propose to be done, action items as a direct result
of your project; \emph{or} specific outcomes already observed. The
project's impact on key stakeholders, whom you've clearly identified.

\hypertarget{license}{%
\section{License}\label{license}}

The template was created for publication in
\href{http://journals.library.iit.edu/index.php/Soremo/}{SoReMo}.

Authors of the reports retain copyright of their work.

This document is created by \href{www.sonjapetrovicstats.com}{Sonja
Petrovic} for \href{www.soremo.org}{SoReMo} and is released under the
same licence as the parent
\href{https://sondzus.github.io/SoReMo/}{folder} on Github.

\hypertarget{acknowledgements}{%
\section{Acknowledgements}\label{acknowledgements}}

Optional section. The Acknowledgements should contain text acknowledging
non-author contributors. Acknowledgements should be brief, and should
not include thanks to anonymous referees and editors or effusive
comments. If applicable, additional grant numbers or supporting agencies
may be acknowledged.

\hypertarget{references}{%
\section{References}\label{references}}

If you are able to knit your own HTML/PDF from Markdown, leave this
section blank. The bibliography will be created from your file
\texttt{mybiblio.bib.} If you are uncomfortable, add your references
here in style as close to BibTex as possible, and the editors will help
with the rest.

\end{document}
